\documentclass{article}
\usepackage[utf8]{inputenc}
\linespread{2}
\title{Senior Research Rough Draft}
\author{skibap }
\date{April 2016}

\begin{document}

\begin{titlepage}
	\centering
	{\scshape\LARGE Eastern Connecticut State University \par}
	\vspace{1cm}
	{\scshape\Large Senior Research Project \\ Rough Draft\par}
	\vspace{1.5cm}
	{\huge\bfseries The Effectiveness of Social Media Spam to Increase Site Traffic\par}
	\vspace{2cm}
	{\Large\itshape Patrick Skiba\par}
	\vfill
	supervised by\par
	Dr.~Garrett \textsc{Dancik}

	\vfill

% Bottom of the page
	{\large \today\par}
\end{titlepage}

\section{Abstract}
\section{Introduction}
\subsection{Significance} Social media has been a popular method of communicating with people for several years now. Despite the efforts of the companies behind these social media services, it does not take long after creating an account to be exposed to someone advertising themself to you via direct messaging or a follow notification. The objective of this research is to measure the effectiveness of spam marketing via social media as well as recording the strength of anti spam techniques used by social media services.

\subsection{Background} Social media spam is on the rise, becoming the preferred communication medium among spammers (Nextgate, 2013) due to the effective security infrastructure that email has developed to block spam emails (Syverson, Paul 2008). Additionally, users of social media sites are far more trusting of the content they see on these sites compared to emails or random web pages (Cao, 2014). Not only is the spam protection on social media weaker than email, but spammers have figured out methods to use the features social media services provide to deliver spam in completely new ways (Cheng, et al., 2014). 
In the past, spammers could only send direct messages to the masses they were trying to reach.  Follow bots, stolen accounts, fake accounts, spam apps, and like-jacking are all feasible due to the features provided by social media platforms (Nextgate, 2013). Follow bots will be the main focus of this research. This involves following a large number of people, thereby triggering a notification on their account, in hopes that their curiosity will lead them to click on a link in the spammer�s profile. 

\section{Methods}
\subsection{Blog}
\subsection{Bot Software}
\subsection{Analytics}
\section{Results}
\section{Discussion}
\newpage
\begin{thebibliography}{9}
\bibitem{Bib1} 
Cao, Qiang. 
\textit{``Understanding and Defending Against Malicious Identities in Online Social 
Networks."} 
Duke. Duke University, 01 Jan 2014.

\bibitem{Bib2} 
 Hu, Xia, and Huan Liu.
 \textit{``Mining Spammers in Social Media: Techniques and Applications.'' }
PAKDD 2014 Tutorial. Arizona State University, 01 Jan. 2014.

\bibitem{Bib3}
Nextgate.
\textit{``2013 State of Social Media Research Report."} 
Nextgate 01 Sept.2013. Web. 3 Mar. 2016.

\bibitem{Bib4}
Syverson, Paul, Somesh Jha, and Xiaolan Zhang.
\textit{``Spamalytics: An Empirical Analysis of Spam Marketing Conversion.''} 
New York: Association for Computing Machinery, Berkeley. International Computer Science Institute, 01 Jan. 2008.

\bibitem{Bib5}
Shin-Ming Cheng, Min-Yuh Day.
\textit{``Technologies and Applications of Artificial Intelligence: 19th International Conference.''}
TAAI 2014, Taipei, Taiwan, 21 November. 2014.
 
 
\end{thebibliography}

\end{document}